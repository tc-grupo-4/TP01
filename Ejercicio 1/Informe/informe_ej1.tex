\documentclass[12pt,a4paper]{article}
\usepackage[utf8]{inputenc}
\usepackage[siunitx,american]{circuitikz}
\usepackage{pgfplots}
\usepackage[margin=0.5in]{geometry}
\usepackage{textcomp}
\usepackage[spanish, es-tabla]{babel}
\usepackage{amsmath}
\usepackage{graphicx}
\usepackage[colorinlistoftodos]{todonotes}
\usepackage{amsmath}
\usepackage{tikz}
\usetikzlibrary{arrows}


\usepackage{parskip}
\usepackage{fancyhdr}
\usepackage{vmargin}
\setmarginsrb{3 cm}{2.5 cm}{3 cm}{2.5 cm}{1 cm}{1.5 cm}{1 cm}{1.5 cm}


\pgfplotsset{compat=1.15}

\begin{document}
\section{Ejercicio I}
\subsection{Analisis del circuito}
En este ejercicio se analizo el circuito \ref{fig:circuito_1}. 

\begin{figure}[ht]                                                       
    \centering\includegraphics[width=0.8\textwidth]{circuito_1.png}
    \caption{Filtro Notch Pasivo}
    \label{fig:circuito_1}
    \end{figure}


En primer lugar, se calculo analíticamente al circuito mediante un método alternativo como es el de cuadripolos para
obtener la función transferencia $H(s)$ que se puede ver en la ecuación \ref{transferencia_1}. Vale aclarar que se tomo la ayuda propuesta por la catedra y se considero que $R_{1}$ = $R_{2}$
= $2R_{3}$, $2C_{1} = 2C_{2} = C_{3}$

\begin{equation} H(s) = \frac{(\frac{S}{1/C_{3} R_{3}})^2 + 1} {(\frac{S}{1/C_{3}R_{3}})^2 + 4\frac{S}{C_{3}R_{3}} + 1}  \label{transferencia_1}\end{equation}

Como se puede observar, la función transferencia describe un filtro Notch. Su frecuencia de corte es
$Wc$. Su expresión se muestra en la ecuación \ref{frecuencia_corte}

\begin{equation} W_{0} =  \frac{1}{C_{3} R_{3}}  \label{frecuencia_corte}\end{equation}

La frecuencia de corte pedida es $10.8k Hz$. Entonces nos queda la relación que se puede ver en la ecuación 
\ref{relacion_RC}.

\begin{equation} R_{3} = \frac{1}{C_{3} 2\pi 10.8k} \label{relacion_RC}\end{equation}

Para obtener la respuesta inpulsiva $h(t)$, se utilzo la antitrasformada de Laplace. Esta resulto ser: 

\begin{equation} h_{t} = asdasd  \end{equation}

    %dirac(t) - 4*w*exp(-2*t*w)*(cosh(3^(1/2)*t*w) - (2*3^(1/2)*sinh(3^(1/2)*t*w))/3)



Volviendo a la relación \ref{relacion_RC} es posible dar valores a la capacitor y asi obtener un valor para las resistencias. Teniendo en cuenta
los valores comerciales disponibles en el pañol, se tomo $C_{3} = 10nF$ por lo que se obtuvo $R_{3}=1.47k\Omega$. Como
no hay disponible una resistencia de ese valor, se utilizo $R_{3}=1.5k\Omega$. Tampoco se encontraron capacitores de $C = 5nF$ 
por lo que $C_{1} = 4.7nF$ y  $C{2} = 4.7nF$. Estos valores se cargaron en LTspice y se obtuvo el bode de la 
figura \ref{fig:bode_ltspice_teorico}.

\begin{figure}[ht]                                                       
    \centering\includegraphics[width=0.8\textwidth]{bode_ltspice_teorico.png}
    \caption{Circuito con los componentes definidos}
    \label{fig:bode_ltspice_teorico}
    \end{figure}

Se puede observar que el comportamiento del bode describe un filtro notch y que la frecuencia de corte se ubica en $11.1kHz$. Si
bien la frecuencia de corte pedida es $10.8kHz$ nos vemos obligados a tomar $11.1kHz$ por motivos de disponibilidad de componentes
en el pañol. Luego las futuras mediciones se comparan respecto al bode obtenido en la figura \ref{fig:bode_ltspice_teorico}. \\

Para poder terminar de caracterizar el sistema hace falta el diagrama de polos y ceros. Los polos y ceros
se obtienen facilmente si reordenamos la funcion trasferencia como se ve continuación:

\begin{center}
    $H(S) = \frac{(S-S_{1})(S-S_{2})}{(S-P_{1})(S-P_{2})S}$  \\
\end{center}

Hay dos ceros:
\begin{center}
    $S_{1}=69743.35691j $
    $S_{2}=-69743.35691j $
    \end{center}

Hay dos polos:

\begin{center}
    $P_{1}=-18687.67616$
    $P_{2}=-260285.7515$
\end{center}

Como se puede ver los dos ceros se encuentran sobre el eje imaginario y los dos polos en el eje real del
semilado negativo


\begin{center}\begin{tikzpicture}
    \draw   (5,0) -- (-5, 0)
            (0,5) -- (0,-5);
    \draw   [red, thick](0,0) -> (0,4.5) node[anchor=north east] {S1};
    \draw   [red, thick](0,0) -> (0,-4.5) node[anchor=north east] {S2};
    \draw   [blue, thick](0,0) -> (-1,0) node[anchor=north east] {P1};
    \draw   [blue, thick](0,0) -> (-2,0) node[anchor=north east] {P2};
    \end{tikzpicture}
    \end{center}

\subsection{Respuesta en frecuencia}
Con los valores de los componentes calculados anteriormente, se diseño una placa en Altium. Su diseño se 
puede ver en la figura \ref{fig:placa_altium}

\begin{figure}[ht]                                                       
    \centering\includegraphics[width=0.8\textwidth]{placa_altium.png}
    \caption{Placa diseñada en Altium}
    \label{fig:placa_altium}
    \end{figure}

Notar que se tuvieron que utilizar dos resistencias en serie de $1.5k\Omega$ para obtener una
resistencia de $5k\Omega$. 

Para medir la respuesta en frecuencia se utilizo una senoide en $V_{in}$ y se midió $V_{out}$
con la ayuda de un osciloscopio. En primer lugar se excito al circuito con una senal 
senoidal de $10V$ a una frecuencia de $11.1kHz$, que es la frecuencia de corte. Se esperaria ver 
una senal totalmente atenuada ya que la senal de entrada esta en la frecuencia de corte. 
Los resultados fueron los siguientes: $V_{in}=9,73V$ y $V_{out}= 0,038V$. Podemos decir 
que el resultado fue satisfactorio ya que se puede considerar que la senal de entrada fue totalmente atenuada.
Si hacemos el calculo de atenuacion esta da $-20\log(\frac{V_{out}}{V_{in}}) = -48,16db$. Esta atenuacion deberia ser
la mas chica cuando se realice el bode completo. Ademas hay que tener en cuenta el ruido. El 
osciloscopio es suceptible al ruido por lo que hay que tenerlo en cuenta. Esto explica porque $V_{out}$
no es cero en la frecuencia de corte del Notch. Lo que se esta midiendo en esta situacion es precticamente ruido ya que 
este tiende a aumentar la amplitud de la senal. \\
Se prosiguio a realizar el bode completo. Para esto se mantuvo una senoide de $10V$ pico a pico
y se fue modificando la frecuencia de esta. Sabiendo de antemano como es la curva que 
describe el bode, se tomaron mas puntos en las áreas mas características del bode. Como lo es el area
cercana a la frecuencia de corte. Los resultados del bode completo se pueden ver en la figura \ref{fig:superpuesto}.

\begin{figure}[ht]                                                       
    \centering\includegraphics[width=0.8\textwidth]{Superpuesto.pdf}
    \caption{Filtro Notch Pasivo}
    \label{fig:superpuesto}
   \end{figure}

Como se puede observar, la figura \ref{fig:superpuesto} tiene superpuesto el bode
de LTspice (curva azul) que se mostró en la figura \ref{fig:bode_ltspice_teorico} y el bode que se obtuvo de
forma experimental (curva naranja). Los resultados son 
sumamente satisfactorios. En el bode de la medicion se puede apreciar la frecuencia de corte y como
el resto de los puntos se asemejan a la curva teorica calculada en LTspice. 



\subsection{Respuesta al escalón}
En esta parte se analizo la respuesta al escalón. En primer lugar se calculo la expresión analitica. Teniendo en cuenta que 
la entreda $X(t)$ es el escalon $U(t)$, que su transformada de Laplace es $\frac{1}{S}$ y que la funcion transferencia es la que 
vimos en la ecuacion \ref{transferencia_1}. La transformada de Laplace de la salida nos queda \ref{transferencia_2}


\begin{equation} Y(S) = \frac{S^2+W_{0}^2}{S^2+4W_{0}S+W_{0}^2} * \frac{1}{S}  \label{transferencia_2}\end{equation}

Si acomodamos un poco esta expresion podemos llegar a: \\

\begin{center}
    $Y(S) = \frac{(S-S_{0})(S+S_{0})}{(S-P_{1})(S-P_{2})S}$  \\
    
    $S_{0}=69743.35691j
    P_{1}=-18687.67616
    P_{2}=-260285.7515$
    \end{center}

Si antitrasformamos nos queda:\\

\begin{center}
    $y(t) = (A\exp{P_{1}t}+B\exp{P_{2}t}+C) * u(t)$

    $A = \frac{P_{1}^2 + W_{0}^2}{(P_{1}-P_{2})*P_{1}} = -1.1547$
    $B = \frac{P_{2}^2 + W_{0}^2}{(P_{2}-P_{1})*P_{2}} =  1.1547$
    $C = \frac{W_{0}^2}{(P_{2}P_{1}} = 1$

    \end{center}

%Si simulamos en LTspice la respuesta al ecalon tomando $C_{3} = 10nF$ y $R_{3}=1.47k\Omega$, nos queda los que observamos en 
%la figura \ref:fig{rta_escalon_teorica}

%\begin{figure}[ht]                                                       
%    \centering\includegraphics[width=0.8\textwidth]{rta_escalon_teorica.png}
%    \caption{Respuesta al Escalón}
%    \label{fig:rta_escalon_teorica}
%    \end{figure}

Las mediciones resultaron ser:

\end{document}